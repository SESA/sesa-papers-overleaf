\label{sec:related}

% we ask questions regarding OS role in energy proportional computation
The questions that our investigation sparked
fall within a wider space of research
on the impact of OS design
and hardware and software policies
on the incessant goal of energy proportional computation in datacenters
%cite energy proportionality pioneer work:
~\cite{energyproportion, warehouse-power}.
Much of this research stems from
the challenges of improving the performance of
network-bound datacenter workloads
like MapReduce~\cite{large-scale-mapreduce}
and in-memory key-value stores~\cite{mica, zygos}
while keeping energy consumption at bay.
These challenges can be attributed to
the complex diurnal trends
that are characteristic of datacenter-level utilization,
whereby idle time is common and must be optimized for
~\cite{hotpower2008, powernap, napsac}
while simultaneously maintaining the ability to support high-utilization peaks and strict latency constraints
~\cite{Dynamo, SmoothOperator, oldi-pegasus, adrenaline, ixcp, rubik, eurosys14, zygos}.
%smoothoperator
%characterize power fragmentation
%design clstering based approach 



There is a wide range of work
that targets energy proportionality
with a focus on designing OS policies and mechanisms for power management.
Most of this work presents hardware level optimizations
that manipulate active power states (P-states) %cite: p states
and idle power states (C-states) %cite: c states
by applying feedback control mechanisms %cite
and relying on activity models. %cite
%p-states: active power states
Dynamo~\cite{Dynamo}, IX Control Plane~\cite{ixcp}, Pegasus~\cite{oldi-pegasus}, Adrenaline~\cite{adrenaline}, and Rubik~\cite{rubik}
present implementations
that use DVFS %cite
and RAPL %cite
to tune power-draw.
%smoothoperator?
The authors of ~\cite{heracles} and ~\cite{PerAppPower} go a step further,
exploring and characterizing the interference of co-located
latency-critical versus best-effort tasks
and high versus low CPU demand tasks
when subject to energy tuning via DVFS and RAPL.
In doing so,
they highlight limitations
in using hardware features alone for power management.
Similarly, the authors of ~\cite{hotpower2008}
identify a need to step away
from relying entirely on hardware solutions
and focusing instead on software optimizations,
such as VM migration controllers for power management of an ensemble of nodes.
%c-states: idle power states



%notable limitations of p-states and c-states
\cite{oldi-study} presents some of the limitations of
restricting power management to hardware tuning,
and \cite{powercap} compares the merits and limitations of
different hardware and software power management techniques.
These results led us to question
the power management of OS-centric and application-centric tasks
during idle and active utilization
and the majority of time spent in between.
The system is often not idle but only slightly busy,
yet system sleep states present a challenge in
resuming high activity in response to sudden bursts
while system polling maintains support for high activity
but degrades energy efficiency.
%Such challenges motivate research
%on hybrid OS solutions for energy proportional datacenter computation.
In response to this paradox,
the authors of \cite{oldi-pegasus} define an iso-latency policy
that aims to maintain energy proportionality
at variable, dynamically shifting activity levels
via OS-level optimizations.

Though research efforts present significant energy savings
from well designed dynamic policies
and carefully chosen static configurations,
we are driven to explore the space beyond current findings,
with a focus on unveiling the role of the OS
in exploiting activity and idleness.
We find that this exploration is timely
given the range of work on optimizing software components,
from NIC driver mechanisms~\cite{flexnic, affinityaccept, network-latency}
to the OS network stack~\cite{mtcp, sandstorm, network-latency}
and dataplane~\cite{ix, ixOS, arrakis},
for energy-efficient large-scale computation.
%INJECT MOTIVATION
Hence we consider an OS-level study
with an interrupt-centric experimental methodology (see \S~\ref{sec:log_collect}) 
for exploring idleness, activity, and the dynamic lapses in between.



%INJECT CONTRIBUTION
This work
started with a motivation to study the impact
of a NIC hardware parameter, ITR-delay,
alongside well studied DVFS and RAPL parameters~\cite{rapl2015, rapl2018},
on energy consumption
of a baremetal libOS in contrast to general purpose Linux.
We wished to quantify the benefits that a libOS,
specialized and optimized for network-bound work,
can have on energy proportionality.
Our efforts concluded with several interesting findings,
one of which is the realization of a potential hybrid solution
for manipulating system idle time in our libOS,
whereby the OS switches from interrupt-driven
%(which assumes clear separation of idle and busy activity)
to polling-driven execution
%(which takes a more intermediary position)
under particular hardware energy configurations.
%add result: less interrupts? more energy saving? better performance?
These findings would not have come to light
were it not for the structure of the libOS
and its particular event-driven execution model~\cite{seda, unikernels}.
%ebbrt
They help us assert that we cannot disregard the OS
as a constituent contributer to application performance and overall cycles-per-instruction (CPI).

%			
%when limitations are reached,
%OS level software optimizations
%--> vm migration
%--> nic controllers
%--> network stacks (libOSes and appliances)
%--> dataplane separation from control plane (ix, arrakis)


%INJECT MOTIVATION
%now the impact of OS path lengths and low level OS behavior becomes primary target for research
%--> hence our motivation for this detailed study of libOS performance vs general purpose linux performance
%--> particularly with emphasis on network and data bound performance

 
