\subsection{Application Heavy}
\label{sec:q4}
\begin{itemize}
\item Q2: Impact of OS in application heavy
\item Q4: How do different tuning impact different OSes?
\end{itemize}

In figures~\ref{fig:metrics_netpipe64}-~\ref{fig:metrics_mcdsilo200}, one can see at the CPU light workloads that while the libOS used drastically fewer instructions than Linux by up to 70\%, its CPI was actually up to 2X worse. This increase in CPI may be explained by the fact that libOS currently only goes into the deepest level of sleep (C7) while the majority of sleep states for linux are at lower level. In both workloads, the libOS' min EPP setting used a fast ITR delay of 2 \micro s, having such frequent interrupts to wake up the core means a C7 sleep state may not be the most optimal decision. It is also possible that the libOS is executing less wasted instructions that tend to be executed with a low CPI. Interestingly, we find changing OS' has an impact on application level CPI, for example, as we move to processing heaver workloads such as nodejs and memcached-silo, we can also see the instruction gap has largely been shrunk by 6\%-10\% and these instructions are noticably more efficient, consuming around 70\% of the cycles of Linux even though for example figure~\ref{fig:busy_mcdsilo200} shows libOS stayed just as busy as Linux for the same offered load of 200K in memcached-silo. Moreover in contrast to the low CPU applications, libOS has a substantial CPI advantage over Linux by up to 30\%. This suggests that, contrary to conventional understanding, OS behavior can have substantial impact on the efficiency of even the time spent in application compute. The libOS model of removing boundary crossing, running application work to completion, and dispatching application code directly from an interrupt, may have a secondary effect on the efficiency of that applciation code that was not previously observed. As expected, when the workload becomes more processing heavy, DVFS and RAPL start to play a more prominent role. We believe the libOS' CPI efficiency contributes to its ability to set DVFS and RAPL at lower levels than Linux tuned and both figures~\ref{fig:joule_mcdsilo200}~\ref{fig:joule_nodejs} show the CPI efficiencies translate to lower overall energy use throughout the workload. Moreover, in contrast to Linux tuned, we find the the libOS is always able have achieve greater reductions in energy than time in both nodejs and memcached-silo for the majority of offered loads, whereas for example, Linux tuned had to sacrifice energy for better tail latency in memcached-silo.
