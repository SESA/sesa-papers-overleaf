\subsection{Other Findings}
\begin{itemize}
\item Another interesting tradeoff can be observed in overview figure~\ref{fig:mcdov}, where tuned Linux can drastically lower its energy consumption by up to 50\% if it is willing to sacrifice most of its 500 \micro s SLA budget. We find this is possible by aggressively setting ITR delay at much higher values of 300-400 \micro s. Using an 600K QPS in memcached as an example (data not shown), we find that compared to default Linux, a tuned Linux with an ITR delay of 300 \micro s can lower its energy use by 44\% through a combination of: 1) 93\% reduction in total interrupts, 2) 30\% reduction in instruction use and last-level cache misses, and 3) a 12X increase in C1E, C3 sleep states, and 2X increase in C7 sleep states. To summarize, tuning Linux with a high ITR delay allows it to go to deeper sleep states and more often, further, the amount of reduced interrupt contents also contributed to less code being executed overall. Moreover, we find that it is able to lower its DVFS at a lower level than tuned Linux for minimum EPP, we hypothesize this is related to the fewer amount of non-application related work that is being done. We do not observe this dramatic of an energy savings in the libOS (only 14\% at 600K QPS), we suspect this effect comes into play at greater efficiences only at much higher QPS for the libOS. There have been a plethora of past work in reducing energy use of workloads such as memcached as it exhibits a diurnal pattern where periods of low utilization exist~\cite{hotpower2008, powernap, napsac}, tuning ITR delay can be an additional lever to further reduce a systems' overall energy use. We also find this trade-off to exist in memcached-silo where in tuned Linux can save energy by up to 47\%, and for the libOS, a smaller fraction of additional energy savings of up to 14\% can be gained.

\item We discovered the \textbf{slow-to-stay-busy (s2sb)} effect which uses low processor frequencies as a way to induce polling in run-to-completion paths. This was discovered by noticing dramatic decreases in number of interrupts (up to 100X) in nodejs as DVFS was set to a very low processor frequency. Moreover, we found the bytes transmitted and received did not differ as dramatically from Linux. Upon closer examination of the libOS' receive path, we found the effect of a low DVFS setting is such that application code directly invoked off of an interrupt is being slowed down. Therefore once the code has slowly returned to the interrupt handler, it is possible that new packets have already arrived from the client and ready to be processed. We believe the polling behavior in libOS is both induced by its low ITR of 4 \micro s and the s2sb effect.

\end{itemize}
