\label{sec:intro}
There are benefits to slowing down network driven workloads.

Two types of slowing down:
\begin{itemize}
    \item Slow down when to process packets from NIC
    \item Slow down processor clock speed
\end{itemize}

How do application behave under two types of network driven workloads:
\begin{itemize}
    \item Close-loop: Increase utilization of machines during diurnal troughs, the system controls the amount of admitted work. Faster time to completion == lower energy
    \item Open-loop: Focused on tail latency combined with SLA objectives
\end{itemize}

What are the types of benefits:
\begin{itemize}
    \item A combination of lower time + lower energy to complete some fixed work + (trade-offs that exist within this combination)
    \item A combination of tail latency + lower energy while meeting SLA objectives + (trade-offs that exist within this combination)
\end{itemize}

How is slowing down beneficial dependent on OS structure (os path length):
\begin{itemize}
    \item Linux - monolithic kernel, interrupt driven
    \item LibOS - unikernel, run-to-completion
\end{itemize}

Further questions and discussions:
\begin{itemize}
    \item Race-to-halt vs slowing down. Right now it is heuristics/black box, can this paper help to say when you should do this and why?
    \item Why is this benefit useful for the system policy designer/os designer? You should do this or that
    \item If we care about performance, why not just poll? Why is polling not enough?
    \item Having different ways to trade off performance+energy is a useful tool to have
\end{itemize}

% There are benefits to slowing down network driven processing.

% Race-to-halt vs slowing down. Right now it is heuristics/black box, can this paper help to say when you should do this and why?

% Two ways to slow down:
% 1) Delay when to process packets, 2)Slow down processor speed

% Two types of network driven workloads + justifying this:
% Close-loop - Netpipe, NodeJS web server
% Open-loop - Memcached, Memcached-Silo

% Types of benefit: 
% Lower time + lower energy+tradeoffs
% Lower tail latency + lower energy + SLA+tradeoffs
% Trade-off in performance-energy
% Energy Performance Product (EPP)?
% Benefit compared to what?

% The benefits differ dependent on OS structure (os path length efficiency):
% Linux
% EbbRT

% Why is this benefit useful for the system policy designer/os designer? You should do this or that**
% Energy Performance Product
% Why is this metric of performance+energy important? To who? 

% If we care about performance, why not just poll? Why is polling not enough?

% What is the combination of performance and energy as an additional metric that is useful?
% Having different ways to trade off performance+energy is a useful tool to have.
% If we can lower performance and energy at the same time that is even better.
