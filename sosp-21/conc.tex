%Lower QPS rates of 50K, 100K, and 200K are targeted here due to its increased computation load per request. Similarly, per QPS rate is run on all three systems 5 times each.

We ported our libOS to run on physical hardware, and wanted  to understand how to set the complex physical settings to optimize both performance and energy used for single application environments.
Given the complexity of Linux's settings, we developed instrumentation for both Linux and our libOS to extract a large set of metrics at every interrupt, and did a study that collected a time series log of the metrics across a sweep of a large set of hardware settings for a variety of simple open and closed loop workloads.
We discuss a number of the insights and implications that our analysis of the data shows or suggests.
We had some strong hypotheses that this study quantified and demonstrated; for example, the impact of short code paths on energy.  
However, a number of the insights, such as the unexpected optimal OS adaptations for some settings, where unexpected, and arose from hundreds of hours exploring interactive 3D visualizations of the data.

We believe that, in addition to the specific insights discussed, the  methodology used to obtain those insights is a valuable contribution.
The insights from this study has led to re-design of aspects of our library OS, which we will pursue in the future.    
Moreover, we believe that it raises interesting broad questions about operating system design and implementation.
Does it make sense to expose controls over not only HW and OS policy, but even OS mechanisms (e.g., run to commpletion) to external controllers?  
How can we do that for OS mechanisms?
Can we further integrate with application requirements (e.g., SLO)  metrics and policies to further improve efficiency?

%Our study started with a desire to understand how a
%baremetal library OS can respond to hardware tuning via
%power and NIC parameters. We wished to statically find
%optimal energy-performance points for our target work-
%loads as they execute within our library OS. Naturally,
%this exploration led us to explore hardware-driven power
%management in Linux as a relevant comparison point.
%Furthermore, the complexity of Linux’s default dynamic
%power management policies let us to explore static con-
%figuration of hardware power settings on Linux as well.
%Consequently, we built an interrupt-centric log collec-
%tion methodology that enables us to collect power and
%workload-specific statistics at the granularity of receive
%and transmit NIC interrupts. After exhaustively explor-
%ing ranges of hardware parameter settings for each of our
%OS/workload, we began an analysis of our plethora of
%fine-grain logs and drew several fascinating observations
%and conclusions about the reciprocal impact of OS soft-
%ware and power management policies and configurations.
%The insights we gather from this study raise interesting
%open questions about operating system design and imple-
%mentation.