%\begin{figure}
%\centering
%\includegraphics[width=0.5\textwidth]{figures/timeline_chart}
%\caption[]{Logical execution timeline for a single application request}
%\label{fig:timeline}
%\end{figure}

\subsection{Mathematical Framework}
\label{sec:model}
%todo: some of this text should be in the intro as part of our contribution.
%An important goal for this work has been to create a useful framework for analyzing and exploring energy and performance impacts accounting for OS behavior.  The request timeline breakdown of Figure~\ref{fig:timeline} was our first step in choosing what to include and abstract from our knowledge of both OS software and the power management literature.  This was both to drive the quantitative study and enable the construction of a model that could be used by ourselves and others to explain and explore alternatives both software and hardware.   

In this section we briefly summarize our work in using our request time-line breakdown (Figure~\ref{fig:timeline}) to develop a mathematical framework that can be used to explain and explore software and hardware effects.  In particular, we show how we model an open loop setting, with an arrival rate of $\lambda$, and explore the impacts of changes in the instruction path length and composition.  A more detailed discussion of the framework and how it can be used can be found in Appendix~\ref{sec:appendix}.   

Assuming a setting where the service time is less than or equal to the time between two requests, we define:
%We begin by defining a breakdown of the service time between two consecutive requests arriving at an arrival rate (in queries/requests per unit time) $\lambda$ as:

$\boxed{\delta t = t_{\text{detect}} + t_{\text{osreq}} + t_{\text{app}} + t_{\text{idlepolicy}} + t_q} = \frac{1}{\lambda}$

where 

$\delta t$ = time between the arrivals of two consecutive requests and the remaining terms directly reflect the time-line components.  
%$t_{\text{detect}}$ = blah, $t_{\text{osreq}}$ = blah, $t_{\text{app}}$ = blah, $t_{\text{idlepolicy}}$ = blah, and $t_q$ = blah.

We group together the three terms that have a clear DVFS dependence and define
$t_{\text{work}}$ as $t_{\text{osreq}} + t_{\text{app}} + t_{\text{idlepolicy}}$.  Excluding detection and any quiescent time, defining $t_{\text{latency}}$ as $t_{\text{detect}} + t_{\text{work}}$ for the total service time or latency for a request.

%Note that: $\delta t = \frac{1}{\lambda}$ or equivalently, $t_q = \left[\frac{1}{\lambda} - t_\text{work} - t_{\text{detect}}\right]^+$ where $[x]^+ = \max(x,0)$.


Similarly the total energy consumed during the inter-arrival time, $\delta t$ is:

$\boxed{E = P_\text{detect} t_{\text{detect}} + P_{\text{work}} \left[t_{\text{osreq}} + t_{\text{app}} + t_{\text{idlepolicy}}\right] + P_q t_q} = P_\text{detect} t_{\text{detect}} + P_{\text{work}} t_{\text{work}} + P_q t_q$

This separation lets us explore independent power regimes for $t_{\text{detect}}$ and $t_q$.  In particular we will be assuming interrupts for detection and a deep sleep state as our idle policy.    

An important aspect of our model is our physically motivated abstraction of a processor's DVFS setting, $\Delta$.  While it is a single value we model its ability to have an independent impact on time (as a possible function of frequency) and power (as a possible function of voltage and frequency).  Specifically, we posit 
%power-law dependence of 
$t_{\text{work}}$ and $P_{\text{work}}$ as follows:

$t_{\text{work}} = A\frac{N_i}{\Delta^{1+\alpha}}$

$P_{\text{work}} = B \Delta^{2+\beta}$

where A, B are constants of proportionality, $N_i$ $=$ the total number of instructions and $\alpha$, $\beta$ are real-valued constants that describe the dependence on DVFS.  This allows us, though $\alpha$ and $\beta$, to explore effects in which different instruction mixes, of the software, are affected in different ways with respect to time and energy by DVFS settings.  

%The core assumption here is that the various terms in figure \ref{timeline} don't overlap. We also assume that $\lambda$ is small enough that the service time is less than the inter-arrival time i.e. $t_{\text{service}} \leq \delta t$. This simple model also applies to the closed-loop case with the restriction that the arrival rate is now dependent on $t_{\text{service}}$. Lastly, this is a qualitative model that ignores stochastic effects (i.e. the full probability distribution of various terms). A quantitative extension of this analysis would cast this problem as a probabilistic graphical model and perform Bayesian inference for the underlying parameters.

% qps = 40*1000 #queries per second
% p_busy_min = 20
% p_static = {
%    'c1':1.5, 
%    'c3':0.5,
%    'c4':0.25,
%    'c7':0,
%    'busy': 10
%}
%chosen_sleep = 'c7'

Figures~\ref{fig:simdesc} and~\ref{fig:simplots} illustrates using the framework to explore the described scenario.  In figure~\ref{fig:simplots}, plot a) shows that when instructions are less affected by frequency one expects to see a vertical structure.  Plots b) and c) show that as the paths are composed of instructions that are more frequency sensitive DVFS changes result in a more curved structure in the energy vs latency.  As expected slowing the instructions starts to affect latency in these instruction mixes.   Additionally in all three plots one sees that delaying interrupts via ITR, for a busy fraction, increases latency with not much improvement in energy.  Remembering that for this configuration deep sleep has been configured to use zero energy.  In section~\ref{exp:mcd} we will examine such a scenario in practice.  
\begin{figure}
\fbox{\tiny \fbox{\begin{minipage}{17em}
The following describes the scenario to be studied mathematically:
\begin{enumerate}[leftmargin=*]
    \item An open loop workload with a load of 40000 queries per second
    \item A processor static Energy consumption rate of 10 Watts and an Energy consumption rate at the lowest DVFS setting of 20 Watts,
    \item an idle policy using a deep sleep state that consumes no energy,
	\item the use of interrupts for detection and examining two ITR scenarios to delay detection -- 5\% and 15\% of the inter-arrival time. 
	\item A work time dominated by the OS processing, examining work time scenarios -- 30\% and 70\% of the inter-arrival time when using the lowest DVFS setting.
	
%	\item for fixed QPS and two different fractions of work time relative the QPS implied arrival rate,
	\item and finally examining the above using three different instruction scenarios where each represents a different sensitivity  of time to DVFS for the code paths:
	\begin{enumerate}
		\item no time sensitivity and power sensitive,
		\item small time sensitivity and power sensitive,
		\item larger time sensitivity and power sensitive
	\end{enumerate}
\end{enumerate}
\end{minipage}}

\fbox{
\begin{minipage}{18em}
$$\lambda = \text{QPS} = 40000 \text{q/s} \rightarrow \frac{1}{\lambda}=25{\mu}s$$

%$f_{\text{itr}}$ is a free parameter between 0 and 1.0 signifying the percentage of time spent in the detection phase. $f_{\text{work}}^{\text{max}}$ is also a free parameter between 0.0 and 1.0 describing the percentage of time spent doing application work at the \textit{lowest DVFS setting}. In other words, the work fraction can at most be $f_{\text{work}}^{\text{max}}$ and when DVFS increases, it decreases.

$$t_{\text{detect}} = \text{itr} = \{0.05*25{\mu}s, 0.15*25{\mu}s\}$$

%$f_{\text{work}}$ is the percentage of time spent working on the application and is a function of DVFS as posited above:

%$f_{\text{work}} = \frac{f_{\text{work}}^{\text{max}}}{\Delta^{1+\alpha}}$

%These percentages can be converted into times based on a chosen arrival rate, $\lambda$ as follows:

$$t_{\text{work}} = \{0.3\frac{25{\mu}s}{\Delta^{1+\alpha}}, 0.7\frac{25{\mu}s}{\Delta^{1+\alpha}}\}$$



Any remaining time is the quiescent time and by construction, $t_q \geq 0$.

$$t_q = (25{\mu}s - t_{\text{detect}} - t_{\text{work}})$$

%Using the definition of latency, we define

\begin{equation}
\bm{t_{\text{\bf{latency}}} = \text{\bf{itr}} + t_{\text{\bf{work}}}}	
\end{equation}

%We also define the three power terms below.

$$P_{\text{work}} = 10W + 20W \Delta^{2+\beta}$$


$$P_q = P_{\text{detect}} = 0$$


%All of this can be put together to calculate the total energy consumed in each inter-arrival period:

\begin{equation}
\bm{E = 0J + (P_{\text{\bf{work}}} t_{\text{\bf{work}}}) + 0J}	
\end{equation}

We will model the three instruction scenarios (a,b,c) using the following: $(\alpha,\beta$) pairs:
$$(-1,-1), (-0.5,-2), (-0.5, -1) $$
We plot $E$ vs $t_{\text{latency}}$ for each pair.

%These calculations can be used to compute numerical values for latency, $t_{\text{latency}}$ and energy, $E$ for various values of detection times $f_{\text{itr}}$, maximum work time $f_{\text{work}}^{\text{max}}$ and DVFS, $\Delta$.
\end{minipage}
}
}
\caption[]{Example analysis scenario. Left box describes the scenario the right shows the mapping to the equations }
\label{fig:simdesc}
\end{figure}

%Figure~\ref{fig:model_values} the mapping of this scenario to the mathematical framework.  
   


\begin{figure}
\centering
\includegraphics[width=.8\columnwidth]{figures/model_plots}
\caption[]{Simulated Plots}
\label{fig:simplots}
\vspace{-0.28in}
\end{figure}

