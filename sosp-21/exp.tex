\label{sec:exp}
The main methodology we used in our study is that of manually setting DVFS and ITR delay values (described in \cref{sec:workflow:dvfs},\cref{sec:workflow:itrdelay}) for both Linux and the library OS.
We refer to these OS setups as \textit{Linux-tuned} and \textit{LibOS-tuned}, respectively, in the figures shown.
Moreover, we conducted this study across two more system configurations across the two OSes.
To better understand the degree of trade-offs in Linux, we also ran experiments on a base configuration which we refer to as (\textit{Linux-default}). In this mode, Linux's interrupt delay and processor speed are both controlled dynamically by its built-in policies~\cite{cpufreq_governor,intelitr}.
For the library OS, we also explored a version of slowing down the processor by replacing network interrupts with a polling loop (whereby no sleep states are used). We refer to this setup as (\textit{LibOS-poll}).

For each experimental run, we gathered some hardware performance counters as well as energy consumption.
Figures~\ref{fig:closed_loop_overview},~\ref{fig:mcd_overview},~\ref{fig:mcdsilo_overview} shows an overview of all the experimental runs gathered across the different applications and their respective loads as listed in table~\ref{table:wrkcfgs}.
For each workload, we break down the trade-offs observed by slowing down in the respective systems into 
measurements of performance (i.e. time for completion of closed-loop workloads and 99\% tail latency for open-loop workloads) and measurements of energy use.
In order to reason about these trade-offs, we use two graphical mechanisms to highlight the differences:
\begin{enumerate}
    \item The \textit{size} of each point is representative of the degree with which interrupt delay is used; the {\larger[1]\textit{larger}} the size, the more interrupt delay is \textit{increased} while the \textit{smaller} the size the more it is \textit{decreased} (i.e. faster IO interrupts).
    \item The \textit{color gradient} of each point represents the degree of slowing down processor speeds; the \textbf{darker} the color the more the processor has been \textit{slowed down} and vice-versa when the color is \textit{lighter}.
\end{enumerate}
Finally, for each of the system configurations studied, the configuration that yields the best performance and lowest energy is indicated with {\larger[4]\textbf{+}} and {\larger[4]\textbf{x}} respectively.
