\label{sec:exp}
The main methodology we used to conduct our study is to manually set DVFS and ITR delay values (described in \cref{sec:workflow:dvfs},\cref{sec:workflow:itrdelay}) for both Linux and the library OS. For each experimental run, we gathered some hardware performance counters and its energy consumption. We refer to these data points as \textit{LibOS-tuned} and \textit{Linux-tuned} in the figures below. Figures~\ref{fig:closed_loop_overview},~\ref{fig:mcd_overview},~\ref{fig:mcdsilo_overview} shows an overview of all the experimental runs gathered across the different applications and their respective loads as listed in table~\ref{table:wrkcfgs}. For each workload, we break down the trade-offs in slowing down in the respective system types in terms of performance (time for closed-loop and 99\% tail latency for open-loop) and their respective energy use. In order to reason about the trade-offs that occur when slowing down the processor speed and interrupt delay, we use two graphical mechanism to highlight the differences: 
\begin{enumerate}
    \item The \textit{size} of each point is representative of the degree with which the interrupt delay used; the {\larger[1]\textit{larger}} the size the more interrupt delay is \textit{increased} while the \textit{smaller} the size the more it is \textit{decreased} (faster IO interrupts).
    \item The \textit{color gradient} of each point represents the degree of slowing down processor speeds; the \textbf{darker} the color the more the processor has been \textit{slowed} and vice-versa when the color is \textit{lighter}.
\end{enumerate}

In total, we conducted this study across four system types across the two OSes. To better understand the degree of trade-offs in Linux, we also ran experiments on a base configuration (\textit{Linux-default}), in this mode, Linux's interrupt delay and processor speed are both controlled dynamically by its built-in policies~\cite{cpufreq_governor,intelitr}. For the library OS, we also explored a version of slowing down the processor by replacing network interrupts with a polling loop (no sleep states used) (\textit{LibOS-poll}). Furthermore, for each of the system type measured, the configuration that yields the best performance and best (lowest) energy are indicated with {\larger[4]\textbf{+}} and {\larger[4]\textbf{x}} respectively.