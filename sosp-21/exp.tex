\label{sec:exp}
Figures~\ref{fig:closed_loop_overview},~\ref{fig:mcd_overview},~\ref{fig:mcdsilo_overview} shows an overview of all the experimental runs gathered across the different applications and their respective loads as listed in table~\ref{table:wrkcfgs}. For each workload, we break down the trade-offs in slowing down in the respective system types in terms of performance (time for closed-loop and 99\% tail latency for open-loop) and their respective energy use. In order to reason about the trade-offs that occur when slowing down the processor speed and interrupt delay, we use two graphical mechanism to highlight the differences: 
\begin{enumerate}
    \item The \textit{size} of each point is representative of the degree with which the interrupt delay used; the {\larger[1]\textit{larger}} the size the more interrupt delay is \textit{increased} while the \textit{smaller} the size the more it is \textit{decreased} (faster network interrupts).
    \item The \textit{color gradient} of each point represents the degree of slowing down processor speeds; the \textbf{darker} the color the more the processor has been \textit{slowed} and vice-versa when the color is \textit{lighter}.
\end{enumerate}

In total, we conducted this study across four system types across the two OSes. For linux and the libOS, the bulk of this work is focused on studying the effects of slowing down processor and interrupts by manually controlling them, and in the figures below, we refer to these data points as \textit{LibOS-tuned} and \textit{Linux-tuned}. To better understand the degree of trade-offs in linux, we also ran experiments on a base configuration\textit{Linux-default}), in this mode, linux's interrupt delay and processor speed are both controlled dynamically by its built-in policies. For the libOS, we also explored a version of slowing down the processor by replacing network interrupts with a polling loop (no sleep states used) (\textit{LibOS-poll}). Furthermore, for each of the system type that we've measured, the configuration that yielded the best performance and best (lowest) energy are indicated with a {\larger[4]\textbf{+}} and {\larger[4]\textbf{x}} respectively.