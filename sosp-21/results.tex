\section{Results}
\label{sec:results}

For every application used in our experiments, we label results from the  three \textit{systems} as:
%In our results below, we present data points for three main systems in each application workload:
\begin{itemize}
    \item \textbf{Linux Default} refers to normal Linux where both the interrupt-delay and DVFS values are set by their dynamic policy.
    \item \textbf{Linux Tuned} and \textbf{Library OS Tuned} refer to Linux and the libOS where the hardware parameters are statically tuned such that EPP is minimized.
\end{itemize}

\subsection{Netpipe}
\label{sec:results:netpipe}
%By tuning ITRDVFS, we achieve a 4x decrease in total time spent on this workload, though that comes at a cost of higher energy consumption rate in Linux tuned.

%This figure also provides context for the interactions that occur between the message size, the protocol stack and the bytes on the wire as message size change. At a large 512 KB message, the network becomes the bottleneck and all three systems start converging at peak throughput, as a larger faction of time is spent on the wire and the system' idle time proportionally increases.  At small to medium sizes more time, per-round trip, is spent in the system code; device driver and protocol processing, thus the load on system's software is proportionally higher.

%At larger message sizes the systems ability to detect and exploit large idle times will be exposed and at small message sizes the systems software's efficiency will be highlighted.  Medium messages sizes will expose a mixture of the systems ability to detect and exploit idle opportunities while also exposing its packet processing efficiency. The less time spent packet processing the greater the opportunity to idle.   

%, thereby illustrating the latency and throughput trade-offs in this application.

%TODO YA
%This figure has the potential to provide context of the different interactions that occur between the workload, protocol stack, and bytes on the wire as message size changes, which illustrates the latency and throughput trade-offs in this application.
%We hypothesize that this result is mainly due to the inefficiencies of the dynamic itr-delay algorithm to adapt to a simple ping-pong workload where a simple static itr-delay value set to a message specific size can result in improved packet processing efficiency. \textbf{as message size changes the workload changes with respective to the protocol stack and bytes on the wire, system tradeoff between latency and throughput changes as size changes}

%% \begin{figure}
%% 	\includegraphics[width=\columnwidth]{asplos2021_figures/netpipe_edp_8192.png}
%% 	\caption{Plot for a fixed message of 8 KB where EDP is minimized.}
%% 	\label{fig:netpipe8192edp}
%% \end{figure}	
%% \begin{figure}
%% 	\includegraphics[width=\columnwidth]{asplos2021_figures/netpipe_combined_barplot_8192.png}
%% 	\caption{Raw data collected from logs for 1 sample run of message size 8 KB normalized to Linux default.}
%% 	\label{fig:netpipe8192bar}
%% \end{figure}

\begin{figure}[t]
  \includegraphics[width=\columnwidth]{osdi_figures/netpipe_tput.pdf}
  \caption{Throughput results across message sizes when tuned for performance.}
  \label{fig:netpipe_tput}
\end{figure}

Figure~\ref{fig:netpipe_tput} illustrates results from a pure throughput perspective across four message sizes (64B up to 512 KB).
By tuning the ITR to a fixed value, we are able to a find a specific hardware configuration for each message size that optimizes overall throughput.
Linux tuned results in a 1.5X to 2X improvement in throughput across message sizes, the libOS further improves the performance by a factor of 1.6X to 3X as compared to Linux. Figure~\ref{fig:netpipe_tput} also shows that as payload size increase, the network becomes the bottleneck and all three systems start converging at peak throughput; this implies that as a larger faction of time is spent on the wire, potentially increasing each systems' idling time. At small to medium payload sizes, there is more of a mix between packet transmission time and time spent in the system code, device driver, and protocol processing.

\subsubsection{General Observations}
From the datapoints listed in Table~\ref{table:eppsum}, Linux tuned outperforms default Linux across all four message sizes in EPP by up to 60\%, further, the LibOS outperforms Linux default across all four message sizes by up to 80\%. We find that in all cases, this is achieved by reducing both time and energy to complete the offered load. We found RAPL to play a minimal role in power tuning for this application given it is not a processor nor memory intensive workload and runs on a single core. In order to observe general trends of hardware tuning effects on Linux and the libOS, we examined the data from a set of low EPP configurations. We found that at smaller message sizes (64 B, 8 KB), ITR plays a pivotal role in reducing EPP given that vast majority of ITR values were less than 10 \micro s, however, there was wide variations in DVFS values for us to make a statement about its usefulness at low message sizes, we suspect the system is largely too lightly loaded for it to play a role. At a message size of 64 KB, the ITR values have largely moved to the 8 - 30 \micro s range, and at 512 KB to the 26 - 38 \micro s range - this is due to packet transmission becoming a larger bottleneck, the larger ITRs used by both Linux tuned and libOS imply a form of artificial packet coalescing is being induced. At these larger message sizes, DVFS plays a more profound role as we see values between \texttt{0xc00 - 0x1500} being consistently applied by Linux tuned and \texttt{0xc00 - 0x1000} being used by the libOS. In addition, the libOS is able to set its DVFS at a value due to its smaller overall codebase, we find that across all message sizes, the instruction usage of libOS is 43\% - 80\% less than Linux and suffered 71\% - 97\% less last-level cache misses. These instruction level efficiences enable the libOS to support the workload while drastically lowering its energy usage through reduced processor frequency; this is also supported by the fact that the libOS's energy usage fell by a greater percentage than time used in the larger message sizes.

\begin{figure}[t]
  \includegraphics[width=\columnwidth]{osdi_figures/netpipe_524288_epp.pdf}
  \caption{Plot for a fixed message of 512 KB where EPP is minimized.}
  \label{fig:netpipe524288epp}
\end{figure}

\begin{figure*}[htb]
\centering	
\begin{minipage}[t]{0.45\textwidth}
\includegraphics[width=\textwidth]{osdi_figures/netpipe_524288_barplot.pdf}
	\caption{Raw data collected from logs for 1 sample run of message size 512 KB normalized to Linux default.}
	\label{fig:netpipe524288barplot}
\end{minipage}
\begin{minipage}[t]{0.45\textwidth}
	\includegraphics[width=\columnwidth]{osdi_figures/netpipe_524288_nonidle_timeline.png}
	\caption{Per-interrupt non-idle ratio for 512 KB message.}
	\label{fig:netpipe524288nonidle}
\end{minipage}
\begin{minipage}[t]{0.45\textwidth}
	\includegraphics[width=\textwidth]{osdi_figures/netpipe_524288_joule_timeline.png}
	\caption{Per-interrupt joules consumed for 512 KB message.}
	\label{fig:netpipe524288joule}
\end{minipage}
\end{figure*}


\subsubsection{512 KB Analysis}
Figure~\ref{fig:netpipe524288epp} shows the joules consumed as a function of time across the length of the workload that sends fixed-size 512KB messages over 5000 round-trips. The line is generated from the detailed per-interrupt log data, the points illustrated are a sub-sampling to improve legibility.  
The slope in this figure is representative of the rate of energy consumption for each system.
Through hardware tuning, Linux achieves 25\% savings in EPP and the libOS achieves 37\% savings, moreover, the rate of energy consumption has also been lowered.

%Figure~\ref{fig:netpipe8192edp} shows the EDP of running our workloads across the three systems upon tuning the aforementioned hardware knobs  towards lower EDP.

%%This figure shows the joules consumed as a function of time across the length of the workload that sends fixed-size 8KB messages over 5000 netpipe round-trips. The line is generate%%d from the detailed per interrupt log data, the points illustrated are a sub-sampling to improve legibility.
%The slope in this figure is representative of the rate of energy consumption for each system.

%By tuning interrupt-delay and DVFS, we achieve a 4x decrease in total time spent on this workload, though that comes at a cost of higher energy consumption rate in Linux tuned.

Figure~\ref{fig:netpipe524288barplot} shows some collected log data gathered from one sample run of this workload and are all normalized against Linux default. Note, the smaller data transmitted in libOS is due to its simpler TCP/IP stack where there are less options used in its TCP/IP header packets. We observe that while the static tuning of Linux and the libOS both reduced the workload in time and used less energy, the number of interrupts was actually 2X higher than default Linux. We attribute this high number of interrupts to be the statically set ITR delay value of 28 \micro s and 26 \micro s for Linux tuned and libOS respectively. This difference is illustrated in figure~\ref{fig:netpipe524288timediff}, where the time difference between every hardware interrupt is shown as function of time for the entire experiment - this figure shows a much wider range of time between interrupts for default Linux. It also shows that tuned Linux's time between every interrupt is roughly concentrated on the ITR delay value that was set, similar to the libOS, except the libOS also exhibits interrupts with wider differences between them. The implication of these diferences are dependent upon the system.

Figure~\ref{fig:netpipe524288nonidle} shows the percentage of time each system was non-idle every 1 ms (this is due to the sampling period we used in this study for reading PMC registers), the non-idle value is derived by taking the ratio between fixed reference cycle data, which measures un-halted core cycles, and the difference between timestamps. From this figure, we can see distinct behaviors between tuned Linux and the libOS where tuned Linux spent most of its time busy to finish the work fast while the libOS was able to both finish the work fast but also take advantage of idling opportunities. The result of this behavior is shown in figure~\ref{fig:netpipe524288joule}, where the two effects of hardware tuning causes the average energy consumption to be lower than Linux default - the 0 joules in figure~\ref{fig:netpipe524288joule} are due to the fact the joule register can only be sampled at a minimum time interval of 1 ms. The efficiency of the libOS and its ability to idle can be seen in figure~\ref{fig:netpipe524288barplot} where the libOS halted into \texttt{C7} sleep state 3X higher than Linux default and the fact the libOS' CPI is 40\% lower. It is also interesting to note tuning Linux resulted in a completely different behavior in terms of idling state where the tuned Linux concentrated all of its idling in the \texttt{C1} state. We hypothesize this is due to the frequent interrupts invoked by the static ITR delay affecting Linux's scheduler. Moreover, the wide range the libOS' time between interrupts could be explained by the high number of \texttt{C7} sleep states as a \texttt{C7} sleep state has exit latencies in the order of hundreds of microseconds while the \texttt{C1} sleep state's exit latency is 2 \micro s.Moreover, even though Linux tuned could not idle as efficiently as the libOS, both systems could set DVFS to a low value of \texttt{0xc00}. We believe this is due to the network heavy nature of NetPIPE when exchanging 512 KB messages, the benefit of this low DVFS value results in a greater percentage of energy savings than time savings for both tuned Linux and the libOS.

%Compared to default Linux, summing up the total number of times c-state was entered, we found tuned Linux idled 23\% less and the libOS idled 96\% less, however the nature of these sleep states counters do not specify 

%This points to greater aggregate workload efficiency coming from fixed lower interrupt delays, 8 $\micro$s and 6 $\micro$s respectively, than what the default algorithm chooses over the life time of the experiment.   This increase in interrupts in 
%tuning Linux results in a 15\% increase in the overall number of instructions executed, however, the temporal savings this translates to yields a significant improvement in energy used and the energy efficiency of the other metrics.  In some sense tuning allow the systems to race to halt to get the work done by  burning energy at a slightly greater rate but using it more efficiently.  

%In our library OS, the number of instructions per interrupt is lower despite a small increase in total interrupt count due to the smaller interrupt-delay value, it also uses dramatically fewer instruction. This results in a considerably higher Bytes transmitted per joule efficiency. A library OS has shorter path lengths due to its dedicated and simplified processing model, further, as application code is executed in the same privileged domain as the device and protocol code and runs in an integrated fashion with the receive interrupt handler. 


%the energy used to satisfy the same work decreases by 3x.
%This speaks to the instruction efficiency of tuning in relation to improving packet processing time.
%Moreover by tuning a library OS, we are able to both reduce instruction count and energy use while supporting the same workload. 

%% \begin{figure}
%% 	\includegraphics[width=\columnwidth]{asplos2021_figures/netpipe_timeline_joules.png}
%% 	\caption{Per interrupt measure of Joule collected for Netpipe 8KB.}
%% 	\label{fig:netpipe8K_joule_tsc}
%% \end{figure}

%Further packet processing efficiency can be seen in figure~\ref{fig:netpipe8K_joule_tsc}, where the per-interrupt number of joules consumed is shown.
%As joules can only be sampled at a minimum time interval of 1 ms, all three systems will exhibit instances in which the time between interrupts is less than 1 ms.
%This is accounted for in the graph by the bottom band of points that report 0 joule energy consumption. These points effectively visualize the temporal interrupt activity that contributes to the 1ms separated Joule readings.  At the scale plotted, it is a hard to compare the interrupt activity captured by the zero points, however, the interrupt bars of Figure~\ref{fig:netpipe8192bar} can be used to infer the relative difference in density of interrupts along the zero line.

%In default Linux, we see two distinct bands of joules consumed, a sparse band showing higher joule consumption, and a dense middle band showing lower joule consumption.
%A simple back-of-the-envelope calculation summing the joules and dividing by the total time gives us a measure of power at 17.29\watt.  This value is very close to the number we have measured when Linux and our library OS are completely idle, with no user applications launched, and they are able to use the deepest sleep state of the processor for the period that their idle behaviour can halt the processor.  

%It should be noted that an OS idling often involves some component of non-halted, time to conduct various  \textit{idle work} -- maintenance and background system work.  Furthermore,  despite being idle once there is application and network activity an OS may have more "idle work" that it must do and the nature of this work maybe stochastic in its costs (eg. uneven garbage collection work).  Despite halting the processor, the OS may use different and less aggressive sleep states given some algorithm it may have for predicting the value and penalty of the sleep states.  While this is true for Linux which is designed to run many applications over many different scenarios, the library OS we use adopts a simple fixed policy of always halting to the deepest sleep state.  

%With the above in mind, we observe fascinating phenomena when considering Figures~\ref{fig:netpipe8K_joule_tsc} and ~\ref{fig:netpipe8Knonidle}. Figure~\ref{fig:netpipe8Knonidle} is derived by taking the ratio between fixed reference cycle data point, which measures un-halted core cycles, and the measured diff between timestamps. 

%Linux default has a bi-modal behavior where it shows a slow steady rate of work that also idles efficiently (close to the best idle power consumption). Unfortunately this drags the total time well beyond what it could have achieved, in other words it neither raced-to-halt globally nor locally. Moreover, when it is doing work it is also more inefficient compared to Linux and the library OS tuned by using more Joules. However, while Linux tuned also managed to race-to-halt globally per interrupt, it was over 50\% busy, which suggests its algorithms averaging its power consumption between the idle consumption and the busy consumption. In contrast, the library OS's shorter path length and simple idle behaviour allow it to both race to halt globally and between interrupts. Thus the library OS is more effectively exploiting the IO nature of the workload to lower the energy consumption. 
%we observe when default Linux is servicing a packet.  
%This power number is exactly the per-package idle power number that we also measured from past experiments.
%This results tells us that the default Linux policy causes Linux it to spend more time idling (the middle band) than actually doing work (the highest band) and is therefore another contributing factor to its EDP. 

%In tuned Linux, there seems to be no idling time at all due to the use of an extremely fast interrupt-delay of 8 $\micro$s, which suggests the benefit of quick turn-around-time for this particular workload.
%Interestingly, while the library OS uses an even smaller interrupt-delay of 6 $\micro$s, it manages to both respond to packets quickly and take advantage of slack time to idle and conserve more energy.
%Moreover, the library OS's instruction efficiency allows it to tune processor frequency at a lower value than tuned Linux.
%The difference between the library OS's idle efficiency and that of tuned Linux is shown in  figure~\ref{fig:netpipe8Knonidle} which uses data from reference cycles collected.
%Reference cycles count the un-halted core cycles that are not affected by processor frequency tuning.
%Therefore, they can be used to infer potential energy savings from idling by observing its ratio with the timestamp counter.
%NOTE FOR YA
%linux is mostly idling
%linux tuned idles least
%lib os tuned idles more or simply executes less overall

%% \begin{figure}
%% 	\includegraphics[width=\columnwidth]{asplos2021_figures/netpipe_nonidle_8192.png}
%% 	\caption{Non-idle ratio for fixed message size at 8 KB}
%% 	\label{fig:netpipe8Knonidle}
%% \end{figure}

%Perhaps a potential benefit of the packet transmission behavior a library OS results in the ability to idle at a more efficient rate. In both systems, when there are no packets to be processed the processors are idling and will be put into some sleep state. These sleep states consists of different levels of energy savings by turning off various processor features.  This figure shows that while Linux default spends 20\% of its time processing and therefore 80\% of its time idling, potentially saving energy by going into sleep states, the increased overall processing time means the base idle cost is still a major contributor to overall energy usage. However we find that by using a library OS, it results in middle ground such that the shortened processing time and the simple packet transmission ability of its device driver results more idle time percentage as compared to Linux tuned. 

%\indent \textbf{summarize results from other message sizes}
%While our study includes detailed experiments for the netpipe workloads at other message sizes, however due to space concerns, we choose to focus on the 8K message size for detailed presentation as it allows one to observe a balance in how the systems behave with respect to busy and idle behavior.  With other message sizes the basic trade-offs remain but in different proportions: 1) at 64 byte messages tuning Linux to minimize EDP reduces the time and energy by 25\% and tuning the library OS by yields a 50\% reduction in time and energy compared to Linux default, 2) at 64KB message size 40\% reduction in time and energy for tuned Linux and 60\% for library OS, and at 512 KB message size 5\% and 10\% respectively.


\subsection{NodeJS}
\label{sec:results:nodejs}
%In the next application, a benchmark sends a constant rate of http requests to a single threaded webserver deployed using the http module running inside a nodejs process. In contrast to the simplified round trip behavior of netpipe, this webserver incurs a heavier processing burden.  Across 
% Even in this regime, tuning hardware knobs matters in both Linux and the library OS
%\begin{table}[]
%\begin{tabular}{| c | c | c | c | c | c |}
%\hline
%           & \multicolumn{4}{c |}{Latency ($\micro$s)} &   \\ \hline
%           & 50\% & 75\% & 90\% & 99\% &  Requests\\ \hline
%\begin{tabular}{@{}c@{}}Linux \\ Default\end{tabular}    & 81   & 82   & 85   & 91   & 365758         \\ \hline
%\begin{tabular}{@{}c@{}}Linux \\ Tuned\end{tabular}    & 74   & 76   & 79   & 85   & 394704         \\ \hline
%\begin{tabular}{@{}c@{}}Library OS\\ Tuned\end{tabular} & 59   & 60   & 61   & 68   & 491765         \\ \hline
%\end{tabular}
%\caption{NodeJS Performance Data}
%\label{tab:nodejs}
%\end{table}
%% across the latencies, variance is mostly similar, 99% tail latency is this, and translates to this throughput

%Table~\ref{tab:nodejs} illustrates the base performance of running a webserver in nodejs. Linux tuned is able to achieve 8\% higher requests than default just by disabling its default DVFS and itr-delay policy and selecting a set of static values that maximizes its performance, while the library OS improves performance by 35\%. With
% for computational work you do have to do, if 
\begin{figure}[t]
	\includegraphics[width=\columnwidth]{osdi_figures/nodejs_edp.pdf}
	\caption{NodeJS EPP plot across three systems}
	\label{fig:nodejs_epp}
\end{figure}
\begin{figure}[t]
	\includegraphics[width=\columnwidth]{osdi_figures/nodejs_barplot.pdf}
	\caption{Collected datapoints of NodeJS across three systems normalized to Linux Default}
	\label{fig:nodejs_bar}
\end{figure}
\begin{figure}[t]
	\includegraphics[width=\columnwidth]{osdi_figures/nodejs_instructions_timeline.png}
	\caption{NodeJS: Per interrupt instruction count.}
	\label{fig:nodejs_instruction}
\end{figure}

\subsubsection{General Observations}
In contrast to NetPIPE, with the move to a processing heavier workload, we found the top minimum EPP points all used ITR delay values of 2 and 4 \micro s for both tuned Linux and the libOS. The efficiency of the libOS over Linux is more dramatic here as the minimum EPP for Linux tuned all required setting DVFS at the higher frequencies (0x1b00-0x1d00), whereas the libOS is still able to lower its processor frequency (0x1900-0x1b00) to conserve additional energy. For the minimum EPP of Linux tuned, DVFS was set at the highest setting along with RAPL while ITR delay was set at lowest value of 2 \micro s. Given these settings, tuning Linux was able to lower its EPP over Linux default by 16\%, the libOS was able to lower its EPP of Linux default by 55\%. In both tuned settings, it was a combination of both savings in energy and time that contributed to the overall EPP.

\subsubsection{Slow to stay busy}
Past researchers have observed there is a tension between using controls like DVFS and RAPL to throttle energy consumption at some budget versus the savings that can be gained by finishing work quick and halting into idle states between the requests for work \cite{} -this latter strategy is often referred to as "race-to-halt" (r2h). In this work, we present a new effect of using processor frequency tuning as a way to "slow-to-stay-busy" s2sb, we found s2sb is only uniquely applicable in the run-to-completion model in the libOS. This was discovered by noticing dramatic decreases in number of interrupts (up to 100X) in NodeJS as DVFS was set to a very low processor frequency. Moreover, we found the bytes transmitted and received did not differ as dramatically from Linux. Upon closer examination of the libOS' receive path, we found the effect of a low DVFS setting is such that on the firing of a receive interrupt, it then has to rewind all the back to the original receive handling code after transmitting a reply, the lowered processor frequency causes this entire process to be slowed down. Moreover, the libOS' receive code is written such that it will poll the NIC for additional packets up to a certain limit, this limit is based upon values used in~\cite{}. Therefore once the code has slowly returned to the receive interrupt handler, it is possible that new packets have already arrived from the client and the libOS begins the process anew. The effect of low DVFS on a run-to-completion based OS is inducing this form of polling, which is based outside of its original intended design.

\subsubsection{NodeJS Analysis}
Similar to NetPIPE, the NodeJS experiment does not induce concurrent load as it is a single stream of serial transactions.  In this case, the client submits requests to a web server written in JavaScript.  This experiment lets us analyze hardware tuning in a application that contains a greater path length in processing every request, in addition the actual data communicated per client-server exchange is small (less than a MTU).  

%In order to present nodejs in a meaningful way with respect to EPP, we selected one sample run of each system hardware configuration with minimzed EDP and scaled the log data to show a fixed workload of up to 100K requests versus using the entire 30 second run.

Given that our load is slightly more realistic, it is natural to ask how the quality of service is impacted by tuning for minimum EPP by considering the measured 99\% tail latency.   For Linux default it is 92 \micro s, Linux tuned at 86 \micro s, and libOS at 66 \micro s.  Thus, static tuning of the hardware parameters yields lower request latency values than the default Linux behaviour of dynamically adjusting the values.

%These latency results also directly %translate to throughput gains given the single connection nature of this benchmark, with an 8\% improvement and 35\% improvement over Linux default by Linux tuned and the the library OS respectively.

% fixed tuning what kind of benefit over base Linux,
% what influence does having shorter code paths on tuning, 
% what does getting to sleep/idleness difference, 
% are you exploiting to sleep, synergistic with instruction efficiency,

Figure~\ref{fig:nodejs_epp} shows that tuning Linux is able to achieve a 8.5\% decrease in both time and energy to service 100K requests, however, the rate with which tuned Linux consumed energy did not change relative to default. Figure~\ref{fig:nodejs_bar} show some surprising differences in the two Linux configurations: 1) Linux default measured 200K total interrupts (100K for each incoming request and 100K for each outgoing transmit complete), however Linux tuned required 300K total interrupts instead (50\% increase) ; 2) Linux default called the \texttt{halt} instruction over 60K times while Linux tuned called the \texttt{halt} instructions 600 times, in both cases 99.99\% of \texttt{halt} states were \texttt{C1}; 3) Linux tuned used fewer instructions and cycles than Linux default and has a 5\% improvement in CPI. Linux's idling function~\cite{linux_idle} states that a \texttt{C1} state has a residency value of 2 \micro s, which is exactly the ITR delay value set for Linux tuned, moreover, figure~\ref{fig:nodejs_timediff} plots the time difference between interrupts and it is possible to see three major bands of interrupt delays, one for receive packet process, transmit completion, and another for the spurious interrupt that the hardware generates at low ITR delay values. These interrupts must be directly interferring with Linux's idling scheduler and is preventing it from going into any sleep states at all, and yet tuning Linux is still able to finish the workload quicker with lower energy usage. This must mean tuned Linux is being placed in a sort of polling mode due to impact of tuning ITR delay, this can be seen in the per-interrupt instruction differences as shown in figure~\ref{fig:nodejs_instruction}, one can see that after every interrupt, tuned Linux manages to be more instruction efficient in terms handling the request. This example also shows cases where it is actually more energy efficient to just poll and finish the work as soon as possible versus taking advantage of any idling. This enforced polling mode might also be the cause of the 5\% improvement in CPI of tuned Linux over its default counterpart.


Figure~\ref{fig:nodejs_epp} also shows tuned libOS is able to achieve a 37\% decrease in energy and 27\% decrease in time to service 100K requests and at a lower rate of energy consumption than Linux default. While the libOS also uses a low ITR delay value of 4 \micro s, figure~\ref{fig:nodejs_bar} shows that its number of interrupts is still similar to Linux default even though its number of \texttt{halt} counts is similar to Linux tuned, meaning the libOS did not have any opportunities to idle. There are two factors contributing to this: 1) the libOS uses shorter packet processing path lengths even though the libOS is running the V8 JavaScript engine executing a HTTP webserver - this is shown in its 30\% CPI improvement over Linux, and 2) the libOS uses a lower DVFS value than Linux and as shown from the findings section, the \textbf{S2SB} effect is in play here by artifically inducing a polling mode inside the libOS.


%We observe that as expected the number of instructions that each system has to execute is dominated by a fixed overhead due likely to the V8 javascript engine executing the javascript webserver.  However, we do see a small drop in instructions executed by the library OS. Given that it does not suffer the increase in interrupt processing we can conclude that the drop is likely from its shorter path lengths.

%Thus it seems we had reduction in work that yields a proportional reduction in total energy consumed.  Looking at figure~\ref{fig:nodejs_bar} we in-fact do see a small drop in the instructions executed and a proportional drop in energy and cycles. 

%While at first it may seem odd that there is a significant increase in interrupts one needs to account for the rate of interrupts that each system is realizing.  Analyzing the log files reveals that the rate of interrupts that Linux Tuned realizes is 40\%  faster than Linux Default.  This faster rate results in more interrupts given that the total time taken by Linux Tuned is only 10\% less than for Linux Default.  On the other hand the library OS rate of interrupts is 27\% faster but it also completes the work in 30\% less time.   These ratios result in the library OS using about the same number of interrupts to do the work as the default while the Linux Tuned spends more interrupts to do the work. 


%Tuning the library OS we find more significant advantages with respect to both reducing energy and time; 37\% and 27\% respectively compared to Linux default.  To understand this consider Figures~\ref{fig:nodejs_joules} and ~\ref{fig:nodejs_non_idle} in context of the Instruction, Energy and Ref. Cycles bars of figure~\ref{fig:nodejs_bar}.  We observe that as expected the number of instructions that each system has to execute is dominated by a fixed overhead due likely to the V8 javascript engine executing the javascript webserver.  However, we do see a small drop in instructions executed by the library OS.  Given that it does not suffer the increase in interrupt processing we can conclude that the drop is likely from its shorter path lengths.

%The drop in Energy and Ref Cycles when put in context of the joules consumed and non-idle time suggest that again while the cpu utilization is considerable compared to netpipe the library OS still manages to extract more opportunities to idle. To gain the ability to more rapidly complete the work per interrupt, given the mean utilization of 85\% between interrupts, versus the close to 100\% utilization for both Linux and Linux tuned,  is converted into more opportunities to reach an efficient idle behaviour between interrupt processing.    

%While there are many more details to the data we have gathered in interesting aspects to discuss, given space constraints we limiting our analysis.  

%27 time reduction 
% 37 %
%Partly it is due to the computationally heavier nature of this application, which for tuned Linux resulted in using over 40\% more interrupts than default as shown in figure~\ref{fig:nodejs_bar}. Intuitively, one would consider an increase in interrupts to correspond with more instructions due to more interrupt handling code and potentially more energy use. However, figure~\ref{fig:nodejs_bar} also shows neither to be the case in tuned Linux versus default, moreover, figure~\ref{fig:nodejs_non_idle} also shows tuned Linux did not get opportunities to idle as well. While tuned Linux was busier, it did not necessarily translate to more redundant work such as packet re-transmissions as figure~\ref{fig:nodejs_bar} shows a slight decrease in total transmitted bytes compared to default. Lastly, this Linux results implies that an for application like nodejs, minimizing overall time spent, or for performance, is the sole metric with which to save energy.

%In both Linux and the library OS tuned, an interrupt-delay of 4 us was selected, which suggests the importance of relying on a small static interrupt-delay value. Figure~\ref{fig:nodejs_edp} shows that tuning the library OS results in a 40\% decrease in time spent while also consuming energy at a reduced rate. While the library OS only used slightly lower instructions as shown in figure~\ref{fig:nodejs_bar}, the spread of energy use per interrupt is more telling of its efficiencies as shown in figure~\ref{fig:nodejs_joules}. This figure indicates that the library OS is consistently using less joules than Linux across the two main bands where work is actively being done. Further, these instruction efficiencies translate into potential processing slack, thereby allowing the library OS idle more as shown in figure~\ref{fig:nodejs_non_idle}.

%\begin{figure}
%	\includegraphics[width=\columnwidth]{osdi_figures/nodejs_joule_timeline.pdf}
%	\caption{NodeJS: Per interrupt measure of Joule}
%	\label{fig:nodejs_joules}
%\end{figure}

%results in aless dramatic difference in overall energy usage. 

%lower instruction -> lower energy, greater efficiency in instruction and potential to sleep more

%ref cycle 


\subsection{Memcached}
\label{sec:results:mcd}
\begin{figure}[t]
	\includegraphics[width=\columnwidth]{asplos2021_figures/mcd_sla.pdf}
	\caption{Performance measure of memcached such that 99\% tail latency < 500 $\micro$s SLA.}
	\label{fig:mcd_sla_tot}
\end{figure}

Figure~\ref{fig:mcd_sla_tot} compares the base performance between Linux default and the two systems when tuned for performance. Given that this is a complicated multicore workload with mixed request rates and the need to satisfy a stringent SLA; tuning hardware parameters in Linux does not produce a noticeable increase in overall peak throughput (but does achieve lower tail latency at low QPS rates). This result is not surprising given the enormous efforts from past researchers to understand and optimize the network paths for workloads such as memcached~\cite{tailatscale, scalingmcdfacebook, workloadanalysisfacebook, ix, ebbrt, farm, 222583}.  The libOS systems optimizations coupled with its baremetal network device driver both contributed to its 2X throughput improvement over Linux.

\subsubsection{General Observations}
Table~\ref{table:eppsum} shows that tuning Linux can lower its EPP by up to 22\% while tuning the libOS can lower its EPP by up to 74\%. For the libOS, we find in all the different cases of QPS loads, that the best ITR value was 2 \micro s; which is the lowest value possible. For tuned Linux, the best ITR values range from 10 \micro s at the lowest offered load up to 30 \micro s at the highest QPS load studied. At low QPS stuned Linux hovered around the midrange for both DVFS (\texttt{0x1500}) and RAPL (\texttt{75}) and at the highest QPS load of 600K, tuning Linux found its minimum EPP with DVFS and RAPL at the highest setting while libOS was able to set ITR, DVFS and RAPL all relatively low. This result is not surprsing given the throughput performances in figure~\ref{fig:mcd_sla_tot} as a QPS of 600K is considered relatively light for the libOS while for Linux it is approaching its peak. Examining the log data we also found that across all the QPS loads, the minimum EPP value for tuned Linux always resulted in increase of its energy use (7.5\%) over Linux default while having a lower 99\% tail latency (22\%). Whereas libOS' minimum EPP consisted of both reductions in 99\% tail latency and energy by up to 56\% and 40\% respectively.

\subsubsection{High ITR Delays}
Another interesting tradeoff can be observed in overview figure~\ref{fig:mcdov}, where tuned Linux can drastically lower its energy consumption by up to 50\% if it is willing to sacrifice most of its 500 \micro s SLA budget. We find this is possible by aggressively setting ITR delay at much higher values of 300-400 \micro s. Using an offered load of 600K QPS as an example, we find that compared to default Linux, a tuned Linux with an ITR delay of 300 \micro s can lower its energy use by 44\% through a combination of: 1) 93\% reduction in total interrupts, 2) 30\% reduction in instruction use and last-level cache misses, and 3) a 12X increase in C1E, C3 sleep states, and 2X increase in C7 sleep states. To summarize, tuning Linux with a high ITR delay allows it to go to deeper sleep states and more often, further, the amount of reduced interrupt contents also contributed to less code being executed overall. Moreover, we find that it is able to lower its DVFS at a lower level than tuned Linux for minimum EPP, we hypothesize this is related to the fewer amount of non-application related work that is being done. We do not observe this dramatic of an energy savings in the libOS (only 14\% at 600K QPS), we suspect this effect comes into play at greater efficiences only at much higher QPS for the libOS. There have been a plethora of past work in reducing energy use of workloads such as memcached as it exhibits a diurnal pattern where periods of low utilization exist~\cite{workloadanalysisfacebook, 10.1145/2024723.2000103,10.1145/2678373.2665718, 10.1145/2806777.2806848}, tuning ITR delay can be an additional lever to further reduce a systems' overall energy use. 

\begin{figure}[t]
	\includegraphics[width=\columnwidth]{osdi_figures/mcd_600000_epp.pdf}
	\caption{Memcached EPP plot across three systems at 600K QPS.}
	\label{fig:mcd_600000_epp}
\end{figure}
\begin{figure}[t]
	\includegraphics[width=\columnwidth]{osdi_figures/mcd_600000_barplot.pdf}
	\caption{Collected datapoints of memcached across three systems normalized to Linux Default.}
	\label{fig:mcd_600000_bar}
\end{figure}
\begin{figure}[t]
	\includegraphics[width=\columnwidth]{osdi_figures/mcd_600000_nonidle_timeline.png}
	\caption{Memcached 600K QPS per-interrupt non-idle ratio.}
	\label{fig:mcd_600000_nonidle}
\end{figure}

\begin{figure}[t]
	\includegraphics[width=\columnwidth]{osdi_figures/mcd_600000_joules_timeline.png}
	\caption{Memcached 600K QPS per-interrupt joule consumption.}
	\label{fig:mcd_600000_joule}
\end{figure}


\subsubsection{600K QPS Analysis}
Figure~\ref{fig:mcd_600000_epp} shows the energy as a function of time across the three systems with minimum EPP. We highlight the 99\% tail latency differences between the three systems as well. In this example, tuned Linux used 5\% higher energy than Linux default while reducing its tail latency by 26\%, libOS used 56 \% less energy and reduced its tail latency by 40\%. Given that memcached is a less compute intensive workload, it is not a surprise that reducing tail latency is important for both tuned Linux and libOS. However, what is surprising is the different combination of ITR, DVFS, and RAPL setting that each system took. The libOS used the lowest ITR value possible (2 \micro s) and set its DVFS/RAPL at the middle ranges, this helped to contribute to its 50\% increase in number of interrupts compared to Linux default. The general efficiency and smaller size of the libOS can be seen in figure~\ref{fig:mcd_600000_bar} as even with the extra interrupt counts, the libOS' instruction count was 65\% less than Linux. This efficiency enables libOS to minimize its 99\% tail latency while using less energy. Even though figure~\ref{fig:mcd_600000_bar} shows that the libOS went to \texttt{C7} sleep state over 2X more than Linux default, we do not believe it actually stayed in those a \texttt{C7} sleep state given its exit latency (hundreds of \micro s) and the dynamic nature of the workload with such a low ITR delay that will constantly waking up the processor. The biggest indication of tuned Linux's difference from default Linux is in the interrupt count, which is 30\% less than Linux default, we attribute this to its ITR delay value of 30 \micro s. This ITR delay seems to shift the ratio of sleep states to focus on \texttt{C1} and \texttt{C1E}, overall it seems tuning Linux allows it to \texttt{halt} more often than Linux default. Figure~\ref{mcd_600000_nonidle} also supports this why showing that tuning Linux spent less time being busy than default. Tuned linux has lower CPI than default, an indication of why it managed to achieve lower tail latency. While tuned Linux was able to be scheduled to sleep more often, it consumed energy at a higher rate than default, could be the exit latencies of more sleep states given the dynamic nature of memcached workload.


%This result demonstrates a distinct tradeoff being made by Linux in terms of how it reached the minimum EPP possible. 
%In contrast to the NAPI polling policy, EbbRT is not limited by a polling budget per receive interrupt, it will process as many receive descriptors as the hardware allows in a synchronous manner, moreover as interrupts are disabled during this entire process, EbbRT potentially could process more packets than Linux which must balance workload with processing budgets.
%Given the base performance of each system, f
%Figure~\ref{fig:mcd600K} illustrates the base energy costs when tuning Linux and library OS to minimize EDP. Similar to nodejs, we also scale the log data to show a fixed workload of 5 million requests for each system and the results we present below is for the 600K QPS rate. We noticed there was a discrepancy in the raw data logs only for Linux tuned in reported EDP values, therefore in figure~\ref{fig:mcd600K} we show the EDP for each run of all systems. It is something we are currently investigating.  However, the results we present regardless show significant differences in the system beyond the displayed variance.  

%While Memcached and the mutilate workload are more represented of a real world cloud service component and load.  It raises interesting challenges from an EDP analysis perspective.  Given that the benchmark generates the requests from a distribution over the request times (gets and sets of particular keys and data) there is no real fixed work rather it is a sample of work.  To this end we plot EDP curves from multiple runs given the observed variance.  While there is variance the optimized EDP shows distinct separation between the systems greater than the variance.  

% fixed tuning what kind of benefit over base Linux,
% what influence does having shorter code paths on tuning, 
% what does getting to sleep/idleness difference, 
% are you exploiting to sleep, synergistic with instruction efficiency,

%% \begin{figure}
%% 	\includegraphics[width=\columnwidth]{asplos2021_figures/mcd_edp_QPS600000.png}
%% 	\caption{Minimum EDP plots when tuned for lowest energy use.}
%% 	\label{fig:mcd600K}
%% \end{figure}

%% \begin{figure}
%% 	\includegraphics[width=\columnwidth]{asplos2021_figures/mcd_combined_barplot_QPS600000.png}
%% 	\caption{Summed up measures from logs and normalized against Linux default.}
%% 	\label{fig:mcdbar600K}
%% \end{figure}

%As stated in the introduction there are dramatic energy saving when comparing the default Linux behavior to statically setting the hardware parameters as is evident in Figure~\ref{fig:mcd600K}.  Perhaps what is most surprising the trade-offs and causality observed in the simpler benchmarks result in a significant win under a far more complex load.  By-passing Linux default behaviour yields similar wins with respect to reducing the utilization between interrupts as indicated in Figure~\ref{fig:mcdnonidle600K}. Specifically, we find the improvement in interrupt processing leads respectively to decreases in nonidle times for both Linux and the Library OS and that these directly are exploited by the idle polocies to dramatically reduce energy.  

%Perhaps one of the most interesting observations comes from the modest QPS of 600k that we have chosen to highlight.   While there are interesting phenomena at higher QPS that the Library OS can support and Linux cannot it is important to note that Tuning has dramatic impacts even in under-loaded senearios.  While attention is usually given to analyzing memcached's performance at high load there is critical headroom for significantly improving the efficiency when servers are operating at lower loads.    

%In the appendix we have include results from our study of Silo-Memcached which integrates an in memory data-base.  This workload induces a more subtle and complex tradeoff between CPU load and IO latency.  Discussion is beyond our space constraints.  
%Although not illustrated we o sweep data reveals that 
%As stated in the introduction 
%The library OS is able to save over 60\% Joules over Linux default by also taking advantage of aggressive interrupt-delay tuning. The drastic difference between the instruction count of the library OS while transmitting slight higher bytes than Linux speaks to the packet processing efficiency. The benefit of a library OS's packet processing efficiency is evident in figure~\ref{fig:mcdnonidle600K} where its non-idle time is roughly half of Linux, another indicator of the ability of a library OS to take advantage of deep sleep states. 




%Our strategy for tuning interrupt-delay is also influenced by previous observations from past work on studying memcached workloads where it can operate in low to medium utilization levels due to diurnal patterns in user traffic~\cite{workloadanalysisfacebook, 10.1145/2024723.2000103}. One way to save energy is to scale a node's energy consumption down under low to medium utilization while satisfying the current SLA. Prior research have tackled this specific problem by using DVFS and RAPL~\cite{10.1145/2678373.2665718, 10.1145/2806777.2806848}. In our study, we take a much wider range of interrupt-delay values (up to 400 $\micro$s) and found that by add aggressive delays in packet processing interrupts, it allows to drastically lower the amount of interrupts to handle and enable better packet processing (shown figure~\ref{fig:mcdbar600K}). By combining this with lower DVFS and RAPL values, it enabled Linux tuned to save between 20\% and 40\% Joules over default. 





%% \begin{figure}
%% 	\includegraphics[width=\columnwidth]{asplos2021_figures/mcd_nonidle_QPS600000.png}
%% 	\caption{Per interrupt measure of non-idle time computed using fixed reference cycles.}
%% 	\label{fig:mcdnonidle600K}
%% \end{figure}



\subsection{Memcached-Silo}
\label{sec:results:mcdsilo}
\begin{figure}
	\includegraphics[width=\columnwidth]{asplos2021_figures/mcdsilo_sla.pdf}
	\caption{Performance measure of memcached-silo such that 99\% tail latency < 500 $\micro$s SLA. Compare default Linux with tuned for performance.}
	\label{fig:mcdsilo_sla_tot}
\end{figure}

\begin{figure}
  \includegraphics[width=\columnwidth]{osdi_figures/mcdsilo_200000_epp.pdf}
  \caption{Memcached-silo 200K QPS min EPP}
  \label{fig:mcdsilo200000_epp}
\end{figure}

\begin{figure}
  \includegraphics[width=\columnwidth]{osdi_figures/mcdsilo_200000_barplot.pdf}
  \caption{Memcached-silo 200K normalized to Linux Default}
  \label{fig:mcdsilo200000_barplot}
\end{figure}

\begin{figure}
  \includegraphics[width=\columnwidth]{osdi_figures/mcdsilo_200000_joules_timeline.png}
  \caption{Memcached-silo 200K QPS energy timeline}
  \label{fig:mcdsilo200000_joule_timeline}
\end{figure}

\begin{figure}
  \includegraphics[width=\columnwidth]{osdi_figures/mcdsilo_200000_nonidle_timeline.png}
  \caption{Memcached-silo 200K QPS non-idle timeline}
  \label{fig:mcdsilo200000_nonidle_timeline}
\end{figure}


%% \begin{figure}
%%   \includegraphics[width=\columnwidth]{asplos2021_figures/mcdsilo_sla.pdf}
%%   \label{fig:mcdsilo_sla}
%%   \caption{Memcached-Silo 99\% tail latency < 500 us SLA}
%% \end{figure}


%% \begin{figure}
%%   \includegraphics[width=\columnwidth]{asplos2021_figures/mcdsilo_edp_aggregated_qps50000.png}
%%   \label{fig:mcdsilo50K}
%% \end{figure}

%% \begin{figure}
%%   \includegraphics[width=\columnwidth]{asplos2021_figures/mcdsilo_edp_aggregated_qps100000.png}
%%   \label{fig:mcdsilo100K}
%% \end{figure}

%% \begin{figure}
%%   \includegraphics[width=\columnwidth]{asplos2021_figures/mcdsilo_edp_aggregated_qps200000.png}
%%   \label{fig:mcdsilo200K}
%% \end{figure}

%\begin{figure}
%	\includegraphics[width=\columnwidth]{asplos2021_figures/netpipe_energy_8192.png}
%\vspace{in}
	%\caption{netpipe 64 KB energy plot}
%	\label{fig:netpipe8K}
%	\vspace{-.2in}
%\end{figure}	
%\begin{figure}%
	%\includegraphics[width=\columnwidth]{asplos2021_figures/netpipe_instructions_8192.png}
	%\caption{netpipe Linux 64K Message }
	%\label{fig:netpipe64K}
%	\vspace{-.2in}
%\end{figure}
%\begin{figure}
	%\includegraphics[width=\columnwidth]{asplos2021_figures/netpipe_instructionsperjoule_8192.png}
	%\label{fig:}
%\end{figure}
%\begin{figure}
%	\includegraphics[width=\columnwidth]{asplos2021_figures/netpipe_txbytes_8192.png}
%	\label{fig:}
%\end{figure}

%\begin{itemize}
    
%\item Tables~\ref{tab:netpipe_linux} and \ref{tab:netpipe_ebbrt} summarizes the tuned results for NetPipe in Linux and EbbRT. This table demonstrates the EDP efficiencies when tuning the aforementioned hardware knobs. Figure~\ref{fig:netpipe_tput} also shows that tuning itr-delay results in improved performance gains in a workload where there exists a tightly bound nature of ping-ponging fixed sized packets between two nodes. 
    
%\item For any of the experimental configurations (Default, Tuned: DVFS, Tuned: DVFS + ITR), as MSG increases, both EDP and TPUT increase. This is expected since for larger MSG sizes, more packets are processed resulting in both a higher time, T and higher energy consumption, E. Throughput is defined to be the total amount of data transferred divided by the total time. Since our network bandwidth is greater than the bandwidth required by the highest message size to transmit, and since NetPipe involves almost no compute-heavy work, the time taken doesn't scale linearly with MSG resulting in higher throughput values. [MAKE CLEAR, REF TO TPUT SCAN]
    
%\item For Linux, tuning DVFS results in significant improvements over the Default case and tuning both ITR and DVFS results in even more EDP gains. These improvement either maintain or significantly improve the workload efficiency metric, which in this case is Throughput.
%    \begin{itemize}
    
%        \item For 64B, tuning DVFS results in an EDP improvement of XX\% and tuning both ITR and DVFS results in an EDP improvement of YY\% over the Default case. These gains not only do not make the TPUT worse (lower) but actually lead to a more efficient queueing system with a TPUT improvement of XX\% (DVFS) and YY\% (ITR + DVFS).
        
%        \item For 8KB, tuning DVFS results in an EDP improvement of XX\% and tuning both ITR and DVFS results in an EDP improvement of YY\% over the Default case. These gains not only do not make the TPUT worse (lower) but actually lead to a more efficient queueing system with a TPUT improvement of XX\% (DVFS) and YY\% (ITR + DVFS).
        
%        \item For 64KB, tuning DVFS results in an EDP improvement of XX\% and tuning both ITR and DVFS results in an EDP improvement of YY\% over the Default case. These gains not only do not make the TPUT worse (lower) but actually lead to a more efficient queueing system with a TPUT improvement of XX\% (DVFS) and YY\% (ITR + DVFS).
        
%        \item For 512KB, tuning DVFS results in an EDP improvement of XX\% and tuning both ITR and DVFS results in an EDP improvement of YY\% over the Default case. TPUT does get a bit worse by tuning DVFS but the effect is negligible at -0.7\%. Furthermore, tuning ITR and DVFS leads to an improved TPUT.
%        \item
%    \end{itemize}
    
 %   \item For EbbRT
  %      \begin{itemize}
   %         \item Running the same configuration as the best (lowest EDP) Linux configuration (BaseLine), results in lower EDP values for all but the 64B case. This strongly demonstrates the advantages of a simpler OS structure in tuning performance. For the 64B case, the OS structure is less relevant given the negligible processing [MAKE CLEAR]. The same gains can be seen in TPUT values [GIVE SOME NUMBERS].
            
    %        \item Tuning and searching across both ITR and DVFS values always leads to savings in EDP ranging from XX\% at the lower end (512KB) to YY\% at the higher end (64B). TPUT also sees improvements from XX\% (KB) to YY\% (KB).
     %   \end{itemize}
%\end{itemize}

%NOTE about additional impact of ITR not just DVFS -> compare to previous results?




%\subsection{NodeJS}


%\begin{figure}
%	\includegraphics[width=\columnwidth]{figures/nodejs_c1_dyn_stat.png}
%	\vspace{-.3in}
%	\caption{EDP: Node.js Linux 1 Connection }
%	\label{fig:netpipe}
%	\vspace{-.2in}
%\end{figure}

%\subsection{Memcached}



%\input{baselines}

%\input{tuning}

