\label{sec:related}

% Table in google docs
% Cloud service focus -- extends the notion of slowing down to include delaying of request detection using interrupt delaying -- versus only considering power management mechanisms.

% OS focused analysis: We focus on understanding how the OS's structure impacts and interacts with the net performance realized and energy consumed.  
% (Gernot work examines the knobs but not the impact of changing the OS)

% General model that frames software behavior that can be adapt as hardware tradeoffs change 

% We use existing interrupt delay feature to delay packet detection when using interrupts.

% Exhaustively document the behaviour of real software

% Closed loop and open loop model 

% we ask questions regarding OS role in energy proportional computation
The questions that our investigation sparked fall within a wider space of research on the impact of OS design and hardware and software policies on the incessant goal of energy proportional computation in datacenters~\cite{energyproportion, warehouse-power, 268014}. Much of this research stems from the challenges of improving the performance of network-bound datacenter workloads like MapReduce~\cite{large-scale-mapreduce} and in-memory key-value stores~\cite{mica, zygos} while keeping energy consumption at bay. These challenges can be attributed to
the complex diurnal trends that are characteristic of datacenter-level utilization, whereby idle time is common and must be optimized for~\cite{hotpower2008, powernap, napsac} while simultaneously maintaining the ability to support high-utilization peaks and strict latency constraints ~\cite{Dynamo, SmoothOperator, oldi-pegasus, adrenaline, rubik, eurosys14, zygos, peafowl, 7425206, 10.1145/2830772.2830779, dreamweaver, dynsleep, udpm}. Our work examines both in-memory key-value stores and its modified version with a heavier processing component as well as closed loop applications. Our incessant goal was not to design another dynamic policy but to gain better insight into the systemic impacts of performance and energy when slowing down these web-centric workloads.

There is a wide range of work that targets energy proportionality with a focus on designing OS policies and mechanisms for power management. Most of this work presents hardware level optimizations that manipulate processor speed mechanisms such as DVFS ~\cite{10.5555/2523721.2523732,10.1145/381677.381702,cpufreq_governor,4273098,packandcap,10.1109/MICRO.2006.8,1598114,10.1145/1629911.1629926,4658633,4343825,10.1109/IGCC.2011.6008552,10.1145/1241601.1241609, slowdownorsleep,4228267, mootaz}, processor power limiting mechanisms such as RAPL~\cite{intel_rapl, heracles, SmoothOperator,oldi-pegasus, Dynamo,PerAppPower,powercap}, and idle power states~\cite{cpuidle_policy,peafowl, udpm,6983037,dreamweaver, pacingtoidle} (c-states) by applying feedback control mechanisms and relying on activity models. The authors of ~\cite{heracles} and ~\cite{PerAppPower} go a step further, exploring and characterizing the interference of co-located latency-critical versus best-effort tasks and high versus low CPU demand tasks when subject to energy tuning via DVFS and RAPL. In doing so, they highlight limitations in using hardware features alone for power management. Similarly, the authors of ~\cite{hotpower2008, 7349225} identify a need to step away from relying entirely on hardware solutions and focusing instead on software optimizations, such as VM migration controllers for power management of an ensemble of nodes. Previous works have advocated for full-system and hardware optimizations for energy~\cite{slowdownorsleep,powernap}, our work builds on their observations and assert that the OS itself plays a big role as well. 

These previous research efforts present significant energy savings from well designed dynamic policies and carefully chosen static configurations, however, we are driven to explore the space beyond current findings with a focus on unveiling the role of the OS in exploiting activity and idleness and also by introducing interrupt delay as an additional knob in this exploration. We find that this exploration is timely given the range of work on optimizing OS paths for performance, from NIC driver mechanisms~\cite{flexnic, affinityaccept, network-latency} to the network stack~\cite{mtcp, sandstorm, network-latency} and the dataplane~\cite{222583, arrakis, EbbRT, shenango, zygos, ix}. Our work was also influenced by previous work in energy efficiency by slowing down both the networking and processor: $\mu$DPM~\cite{udpm} is a application-level policy for memcached to delay request processing and maximize idle periods where deep sleep states can then be utilized, in ~\cite{10.5555/2338816.2338822} the authors combined bandwidth limiting in Cray clusters and scaling processor frequency to reduce energy use of HPC applications. In contrast to $\mu$DPM, we use a hardware register on the NIC to induce interrupt delays and this mechanism can be commonly found in commercial NICs. Lastly, in contrast to all of the previous works, we are the first to conduct such an in-depth study in with a bare-metal specialized OS stack.

%Our work was also motivated by prior research in designing policies using DVFS, request delaying, and c-states to save energy.

%Slowing down Request delaying along with 

%c-states: idle power states
% Dynamo~\cite{Dynamo}, Pegasus~\cite{oldi-pegasus}, Adrenaline~\cite{adrenaline}, and Rubik~\cite{rubik} use the above hardware features to tune power-draw across various applications.

%notable limitations of p-states and c-states
%\cite{oldi-study} presents some of the limitations of restricting power management to hardware tuning, and \cite{powercap} compares the merits and limitations of different hardware and software power management techniques. These results led us to study the power management of OS-centric and application-centric tasks during idle and active utilization and the time spent in between. The system is often not idle but only slightly busy, yet system sleep states present a challenge in resuming high activity in response to sudden bursts while system polling maintains support for high activity but degrades energy efficiency. In response to this, the authors of \cite{oldi-pegasus} define an iso-latency policy that aims to maintain energy proportionality at variable, dynamically shifting activity levels via OS-level optimizations.







%Hence we consider an OS-level study with an interrupt-centric experimental methodology for exploring idleness, activity, and the dynamic lapses in between.
%INJECT MOTIVATION




% %INJECT CONTRIBUTION
% This work
% started with a motivation to study the impact
% of a NIC hardware parameter, ITR-delay,
% alongside well studied DVFS and RAPL parameters~\cite{rapl2015, rapl2018},
% on energy consumption
% of a baremetal libOS in contrast to general purpose Linux.
% We wished to quantify the benefits that a libOS,
% specialized and optimized for network-bound work,
% can have on energy proportionality.
% Our efforts concluded with several interesting findings,
% one of which is the realization of a potential hybrid solution
% for manipulating system idle time in our libOS,
% whereby the OS switches from interrupt-driven
% %(which assumes clear separation of idle and busy activity)
% to polling-driven execution
% %(which takes a more intermediary position)
% under particular hardware energy configurations.
% %add result: less interrupts? more energy saving? better performance?
% These findings would not have come to light
% were it not for the structure of the libOS
% and its particular event-driven execution model~\cite{seda, unikernels}.
% %ebbrt
% They help us assert that we cannot disregard the OS
% as a constituent contributer to application performance and overall cycles-per-instruction (CPI).

%			
%when limitations are reached,
%OS level software optimizations
%--> vm migration
%--> nic controllers
%--> network stacks (libOSes and appliances)
%--> dataplane separation from control plane (ix, arrakis)


%INJECT MOTIVATION
%now the impact of OS path lengths and low level OS behavior becomes primary target for research
%--> hence our motivation for this detailed study of libOS performance vs general purpose linux performance
%--> particularly with emphasis on network and data bound performance

 
